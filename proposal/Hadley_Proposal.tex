% based on the CVPR template provided by Ming-Ming Cheng (https://github.com/MCG-NKU/CVPR_Template)
% modified and extended by Stefan Roth (stefan.roth@NOSPAMtu-darmstadt.de)

\documentclass[10pt,twocolumn,letterpaper]{article}

%%%%%%%%% PAPER TYPE
\usepackage[pagenumbers]{cvpr}

% Include other packages here, before hyperref.
\usepackage{graphicx}
\usepackage{amsmath}
\usepackage{amssymb}
\usepackage{booktabs}

% If you comment hyperref and then uncomment it, you should delete
% Hadley_Proposal.aux before re-running LaTeX.
% (Or just hit 'q' on the first LaTeX run, let it finish, and you
%  should be clear).
\usepackage[pagebackref,breaklinks,colorlinks]{hyperref}

% Support for easy cross-referencing
\usepackage[capitalize]{cleveref}
\crefname{section}{Sec.}{Secs.}
\Crefname{section}{Section}{Sections}
\Crefname{table}{Table}{Tables}
\crefname{table}{Tab.}{Tabs.}

\begin{document}

%%%%%%%%% TITLE
\title{CS 153 Project Proposal\\
Mapping the density of flowers on California buckwheat plants
}

\author{Alex Hadley\\
Harvey Mudd College\\
Claremont, CA\\
{\tt\small ahadley@hmc.edu}
}
\maketitle

% %%%%%%%%% ABSTRACT
% \begin{abstract}
%    The ABSTRACT is to be in fully justified italicized text, at the top of the left-hand column, below the author and affiliation information.
%    Use the word ``Abstract'' as the title, in 12-point Times, boldface type, centered relative to the column, initially capitalized.
%    The abstract is to be in 10-point, single-spaced type.
%    Leave two blank lines after the Abstract, then begin the main text.
%    Look at previous CVPR abstracts to get a feel for style and length.
% \end{abstract}

%%%%%%%%% BODY TEXT
\section{Motivation}

Clearly explain in your own words the overarching motivation of the project from a scientific or engineering perspective. What do you hope your project will do when you are done? This doesn't need to be monumental or reinvent the project description you were provided, it's just helpful to get practice in articulating research goals.

The motivation for this project is\dots

Proposed by Professor Matina Donaldson-Matasci at Harvey Mudd College.

Citation example~\cite{Alpher03}.

\section{Methods}

Present a proposed method of approach. This should be grounded in current practices in computer vision; you do not need to reinvent everything from scratch, but rather figure out what existing tools and components make sense to apply to your project and refine to provide your particular implementation.

Annotation

\section{Measuring success}

Present a plan to measure the success of your efforts. This should include a description of the data you have been provided and how you expect to use it, and how you will quantify the behaviour of your method in the context of your project goal.

\section{Challenges}

Explain any potential challenges you foresee, constraints you expect you might need to introduce to simplify the problem and make it more achievable, or, if appropriate, present any hypotheses (with analysis from supporting references) you have on your expected outcomes.

%%%%%%%%% REFERENCES
{\small
\bibliographystyle{ieee_fullname}
\bibliography{egbib}
}

\end{document}
